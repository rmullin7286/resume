%%%%%%%%%%%%%%%%%%%%%%%%%%%%%%%%%%%%%%%%%
% Medium Length Professional CV
% LaTeX Template
% Version 3.0 (December 17, 2022)
%
% This template originates from:
% https://www.LaTeXTemplates.com
%
% Author:
% Vel (vel@latextemplates.com)
%
% Original author:
% Trey Hunner (http://www.treyhunner.com/)
%
% License:
% CC BY-NC-SA 4.0 (https://creativecommons.org/licenses/by-nc-sa/4.0/)
%
%%%%%%%%%%%%%%%%%%%%%%%%%%%%%%%%%%%%%%%%%

%----------------------------------------------------------------------------------------
%	PACKAGES AND OTHER DOCUMENT CONFIGURATIONS
%----------------------------------------------------------------------------------------

\documentclass[
	%a4paper, % Uncomment for A4 paper size (default is US letter)
	11pt, % Default font size, can use 10pt, 11pt or 12pt
]{resume} % Use the resume class

\usepackage{ebgaramond} % Use the EB Garamond font
\usepackage{hyperref}

\hypersetup{colorlinks=true,linkcolor=blue,urlcolor=blue}

%------------------------------------------------

\name{Ryan Mullin} % Your name to appear at the top

% You can use the \address command up to 3 times for 3 different addresses or pieces of contact information
% Any new lines (\\) you use in the \address commands will be converted to symbols, so each address will appear as a single line.

\address{10919 113th Ct. NE \\ Apt. F305 \\ Kirkland, Washington 98033} % Main address

\address{(509)~$\cdot$~860~$\cdot$~5937 \\ ryan.mullin12@gmail.com} % Contact information
\address{Document source code available on \href{https://github.com/rmullin7286/resume}{Github}}

%----------------------------------------------------------------------------------------

\begin{document}

%----------------------------------------------------------------------------------------
%	WORK EXPERIENCE SECTION
%----------------------------------------------------------------------------------------
\begin{rSection}{Experience}

   \begin{rSubsection}{Apptio, an IBM Company}{May 2019 - Present}{Software Development Engineer II}{Bellevue, WA}
        \item Enhanced calculation performance for Apptio's flagship platform, managing billions in IT spend data for 1800+ companies.
        \item Delivered valuable features in a 1.5 million line monolith Java codebase, as well as greenfield distributed systems in Python, Go, and Rust.
        \item Optimized JVM memory allocations, utilized graph-based data dependency analysis, and improved parallel execution for large-scale IT spend reports.
        \item Solved difficult data consistency issues, race conditions and deadlocks in a highly parallelized, distributed environment
        \item Used CDK and Terraform to manage infrastructure for internal and customer facing services in AWS
        \item Built CI/CD pipelines for Kubernetes and AWS projects using Github Actions and ArgoCD
        \item Leveraged AWS Step Functions, EC2, S3, RDS, Lambda, Athena, and DynamoDB to create scalable systems.
        \item Used Step Functions and Lambda to create our internal system for performance regression testing, used every release.
        \item Designed and implemented a service to ingest over 100M+ calculation analytics datapoints daily across
            our application fleet, used to optimize computations, prune customer data, and provide actionable insights for customers using the Calc Profiler UI.
        \item Passionate about language theory and its applications in software. Ran workshops on functional
            programming in modern Java, created developer guides for advanced features in various languages, 
            and am currently leading the adoption of Rust into our team's tech stack.
   \end{rSubsection}

    \begin{rSubsection}{WSU Tree Fruit Research Center}{June 2017 - August 2018}{Hardware/Software developer}{Wenatchee, WA}
        \item Designed and developed field data loggers to collect luminosity and temperature metrics for apple orchards
            with remote data upload over 3G.
        \item Developed custom hardware drivers for GPIO based modules using C/C++
        \item Wrote image manipulation algorithms with OpenCV to analyze crop cross section scans, published in a
            university thesis.
        \item Our research team made the Cosmic Crisp apple.
    \end{rSubsection}

\end{rSection}


%----------------------------------------------------------------------------------------
%	EDUCATION SECTION
%----------------------------------------------------------------------------------------

\begin{rSection}{Education}
	
	\textbf{Washington State University} \hfill \textit{December 2019} \\ 
    B.S. in Computer Science \\
    3.8 GPA
	
\end{rSection}

%----------------------------------------------------------------------------------------
%	TECHNICAL STRENGTHS SECTION
%----------------------------------------------------------------------------------------

\begin{rSection}{Technical Strengths}

	\begin{tabular}{@{} >{\bfseries}l @{\hspace{6ex}} p{12cm} l @{}}
	    Programming Languages & Java, Kotlin, Rust, Python, Haskell, C, C++, C\#, F\#, Javascript, Typescript, Bash, SQL, Lua \\
		Frameworks & CDK, Terraform, React, Material UI, Dropwizard, GWT, Hibernate, JUnit, PyTest, LocalStack \\
		AWS Technologies & Step Functions, EC2, RDS, DynamoDB, Lambda, Api Gateway, Kinesis Datastream, S3, Athena, DocumentDB \\
		Tools & Kubernetes, Docker, ArgoCD, Github Enterprise, TeamCity
    \end{tabular}
\end{rSection}

\end{document}
